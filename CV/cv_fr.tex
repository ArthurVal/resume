% FortySecondsCV LaTeX template
% Copyright © 2019-2020 René Wirnata <rene.wirnata@pandascience.net>
% Licensed under the 3-Clause BSD License. See LICENSE file for details.
%
% Please visit https://github.com/PandaScience/FortySecondsCV for the most
% recent version! For bugs or feature requests, please open a new issue on
% github.
%
% Contributors
% ------------
% * ifokkema
% * Bertbk
% * Hespe
% * esben
%
% Attributions
% ------------
% * fortysecondscv is based on the twentysecondcv class by Carmine Spagnuolo
%   (cspagnuolo@unisa.it), released under the MIT license and available under
%   https://github.com/spagnuolocarmine/TwentySecondsCurriculumVitae-LaTex
% * further attributions are indicated immediately before corresponding code


%-------------------------------------------------------------------------------
%                             ADDITIONAL PACKAGES
%-------------------------------------------------------------------------------
\documentclass[
	a4paper,
	% showframes,
	% vline=2.2em,
	% maincolor=cvgreen,
	% sidecolor=gray!50,
	% sectioncolor=red,
	% subsectioncolor=orange,
	% itemtextcolor=black!80,
	% sidebarwidth=0.4\paperwidth,
	% topbottommargin=0.03\paperheight,
	% leftrightmargin=20pt,
	% profilepicsize=4.5cm,
	% profilepicborderwidth=3.5pt,
	% profilepicstyle=profilecircle,
	% profilepiczoom=1.0,
	% profilepicxshift=0mm,
	% profilepicyshift=0mm,
	% profilepicrounding=1.0cm,
	% logowidth=4.5cm,
	% logospace=5mm,
	% logoposition=before,
]{fortysecondscv}

% improve word spacing and hyphenation
\usepackage{microtype}
\usepackage{ragged2e}

% uncomment in case you don't want any hyphenation
% \usepackage[none]{hyphenat}

% take care of proper font encoding
\ifxetexorluatex
	\usepackage{fontspec}
	\defaultfontfeatures{Ligatures=TeX}
	\newfontfamily\headingfont[Path = fonts/]{segoeuib.ttf} % local font
\else
	\usepackage[utf8]{inputenc}
	\usepackage[T1]{fontenc}
\fi

% \usepackage[sfdefault]{noto} % use noto google font
\usepackage[sfdefault]{ClearSans}

% enable mathematical syntax for some symbols like \varnothing
\usepackage{amssymb}

% bubble diagram configuration
\usepackage{smartdiagram}
\smartdiagramset{
	% default font size is \large, so adjust to harmonize with sidebar layout
	bubble center node font = \footnotesize,
	bubble node font = \footnotesize,
	% default: 4cm/2.5cm; make minimum diameter relative to sidebar size
	bubble center node size = 0.4\sidebartextwidth,
	bubble node size = 0.25\sidebartextwidth,
	distance center/other bubbles = 1.5em,
	% set center bubble color
	bubble center node color = maincolor!70,
	% define the list of colors usable in the diagram
	set color list = {maincolor!10, maincolor!40,
	maincolor!20, maincolor!60, maincolor!35},
	% sets the opacity at which the bubbles are shown
	bubble fill opacity = 0.8,
}


\usepackage{enumitem}
\newcommand{\cea}{\href{http://www.cea.fr/}{CEA}}
\newcommand{\ros}{\href{http://www.ros.org/}{R.O.S.}}
\newcommand{\ice}{\href{https://zeroc.com/products/ice}{ICE}}
\newcommand{\tango}{\href{https://www.tango-controls.org/}{TANGO}}
%-------------------------------------------------------------------------------
%                            PERSONAL INFORMATION
%-------------------------------------------------------------------------------
%% mandatory information
% your name
\cvname{Arthur Valiente}
% job title/career
\cvjobtitle{Ingénieur Logiciel Embarqué,\\[0.2em] Ingénieur Robotique}

%% optional information
% profile picture
% \cvprofilepic{pics/profile.png}
% logo picture
% \cvlogopic{pics/logo_txt.png}

% NOTE: ordering in sidebar will mimic the following order
% date of birth
% \cvbirthday{\today}
% short address/location, use \newline if more than 1 line is required
\cvaddress{50 rue Pierre Sémard,\newline 38000 Grenoble}
% phone number
\cvphone{+33.6.77.20.10.54}
% personal website
% \cvsite{https://pandascience.net}
% email address
\cvmail{valiente.arthur@gmail.com}
% pgp key
% \cvkey{4096R/FF00FF00}{0xAABBCCDDFF00FF00}
% any other custom entry
% \cvcustomdata{\faGithub}{https://github.com/ArthurVal}

%-------------------------------------------------------------------------------
%                              SIDEBAR 1st PAGE
%-------------------------------------------------------------------------------
% add more profile sections to sidebar on first page
\addtofrontsidebar{
	% include gosquare national flags from https://github.com/gosquared/flags;
	% naming according to ISO 3166-1 alpha-2 country codes
	\graphicspath{{pics/flags/}}

	% social network accounts incl. proper hyperlinks
	\profilesection{Réseaux sociaux}
		\begin{icontable}{2.5em}{1em}
			\social{\faLinkedin}
        {https://www.linkedin.com/in/arthur-valiente-1866937b/}
				{/arthur-valiente-1866937b}
			\social{\faGithub}
				{https://github.com/ArthurVal}
				{/ArthurVal}
		\end{icontable}

	\profilesection{Langues}
		\pointskill{\flag{FR.png}}{Français}{5}
		\pointskill{\flag{GB.png}}{Anglais}{4}
        \pointskill{\flag{ES.png} \flag{JP.png}} {Espagnol/Japonais}{1}
        % \pointskill{\flag{}}{Japonais}{1}

  \profilesection{Compétences}
    \chartlabel{Aperçu}
  	\begin{figure}\centering
  		\smartdiagram[bubble diagram]{
  			\textcolor{white}{\large \textbf{Ingénieur}} \\
  			\textcolor{white}{\large \textbf{Logiciel}}\\ % center bubble
  			\textcolor{white}{\large \textbf{Robotique}}, % center bubble
  			\textcolor{black!90}{\large Systèmes}\\
  			\textcolor{black!90}{\large Distribués}\\
  			\textcolor{black!90}{\ros, \ice}\\
  			\textcolor{black!90}{\tango},
  			\textcolor{black!90}{\large \faLinux}\\
  			\textcolor{black!90}{\large Linux}\\        
        \textcolor{black!90}{\large
          \href{https://www.yoctoproject.org/}{Yocto}},
  			\textcolor{black!90}{\large C/C++}\\
  			\textcolor{black!90}{\large Python}\\
  			\textcolor{black!90}{\large Bash},
        \textcolor{black!90}{\large UML -
          \href{https://fr.wikipedia.org/wiki/Programmation_orientée_objet}{OOP}}\\
        \textcolor{black!90}{Design Pattern},
        \textcolor{black!90}{\large SOC}\\
        \textcolor{black!90}{ARM - DSP} }
  	\end{figure}
    
    \chartlabel{Connaissances - Logiciel/Framework}\\
  	\begin{figure}\centering    
      \barskill[2ex]{}{\large C++ $\bullet$ Python $\bullet$ GIT $\bullet$ CMake }{90}
      \barskill[2ex]{}{\large Bash $\bullet$ Emacs $\bullet$ \ros }{80}
      \barskill[2ex]{}{\large Matlab $\bullet$ Simulink $\bullet$ Make }{40}
      \barskill[2ex]{}{\large \LaTeX }{20}
  	\end{figure}
}


%-------------------------------------------------------------------------------
%                              SIDEBAR 2nd PAGE
%-------------------------------------------------------------------------------
\addtobacksidebar{

}



%-------------------------------------------------------------------------------
%                         TABLE ENTRIES RIGHT COLUMN
%-------------------------------------------------------------------------------
\begin{document}

\makefrontsidebar

\cvsection{Formation}
\begin{cvtable}[1.5]
	\cvitem{2012 -- 2015}{Ingénieur : Génie Électrique et Automatique}{\href{http://www.enseeiht.fr/fr/index.html}{INP-ENSEEIHT}}
  {Commande, Décision et Informatique des Systèmes Critiques\newline
   Développement des Systèmes Informatiques Critiques}
	\cvitem{2010 -- 2012}{D.U.T. Mesures Physiques}{\href{http://iut-meph.ups-tlse.fr/}{I.U.T Paul Sabatier, Toulouse}}
  {Techniques Instrumentales}
	\cvitem{2010}{Baccalauréat S - Science de l'Ingénieur}{\href{http://alexis-monteil.entmip.fr/}{Lycée Alexis Monteil, Rodez}}
  { }  
\end{cvtable}

\cvsection{Expériences professionnelles}
  

\begin{cvtable}[2]
  \cvitem{Janv 2021}{CDI : Ingénieur informatique}
  {Capgemini-DEMS, Toulouse}
  {
    \textbf{Projet de Recherche et Innovation} : Product Owner du Work Package
    navigation autonome d'un bateau
  }
  \cvitem{Mars 2020 -- Juin 2020}{CDD : Ingénieur informatique}
  {\href{https://www.polytechnique.edu/fr/le-laboratoire-leprince-ringuet-llr}{École Polytechnique, Palaiseau}}
  {
    \textbf{\href{http://polywww.in2p3.fr/-cms-45-?lang=fr}{Projet CMS}}:
    \textbf{Rédaction de la spécification} du robot en charge du test
    automatique des détecteurs HGCROC.
  }
  
  \cvitem{Juil 2018 -- Déce 2019}
  {CDD : Ingénieur de Recherche}
  {\cea-DRF-IRFU, Saclay}
  {
    \textbf{Responsable du banc de test distribué} du logiciel embarqué de la
    caméra ECLAIRs (Détection de sursauts gamma):\newline
    $\textbullet$ \href{http://www.svom.fr/}{Projet SVOM} -- Collaboration \cea\
    - \href{http://www.irap.omp.eu/}{IRAP} - \href{https://cnes.fr/fr}{CNES} - CNSA\newline
    $\textbullet$ Développement de l'infrastructure de test distribuée (2x Zedboard, PC,
    plateforme embarqué ECLAIRs) (\textbf{C++}, \textbf{Python})\newline
    $\textbullet$ Création/rédaction des scripts de tests unitaires
    (\textbf{Python})\newline
    $\textbullet$ \textbf{PUS} - \textbf{Spacewire} - Middleware
    \href{https://zeroc.com/products/ice}{\textbf{ICE}}
  }
  
  \cvitem{Janv 2017 -- Déce 2017}
  {CDD : Ingénieur de Recherche}
  {\cea-DRT-LIST, Saclay}
  {
    \textbf{Responsable logiciel pour applications robotiques}:\newline
    $\textbullet$ Co-responsable développement logiciel (Git Admin - \ros)\newline
    $\textbullet$ SLAM hybride topologique pour robot mobile (Lidar / IMU / Caméra RBG-D
    / ...) (\textbf{C++})\newline
    $\textbullet$ \textbf{\href{https://www.designspot.fr/portfolio/face/}{Projet
        FACE}}: En charge de la création du prototype de démonstration multi
    \href{https://en.wikipedia.org/wiki/System_on_a_chip}{System-On-Chip}
    distribué (Renesas R-Car H3, Nvidia Tx2, Kalray MPPA Bostan) pour la voiture
    autonome\newline 
    $\textbullet$ Développement de 'layers' Linux embarqué Yocto pour R-Car H3
  }
  
	\cvitem{Sept 2015 -- Déce 2015}
  {CDD : Ingénieur d'Étude}
  {\href{https://www.irit.fr/?lang=fr}{UPS-IRIT, Toulouse}}
  {
    \textbf{Développement démonstration robotique} sur robots
    \href{http://www.willowgarage.com/pages/pr2/overview}{PR2} et
    \href{https://spectrum.ieee.org/automaton/robotics/humanoids/aldebaran-robotics-introduces-romeo-finally}{ROMEO}:\newline
    $\textbullet$ Projet
    \href{http://www.agence-nationale-recherche.fr/Project-ANR-12-CORD-0003}{ANR
      RIDDLE} -- Collaboration CNRS-LAAS, UPS-IRIT, Aldebaran Robotics, CHU
    Toulouse, Magelium\newline
    $\textbullet$ Importation d'algorithmes de détection d'objets et d'intéractions (vision
    et vocal) sur robot \href{https://spectrum.ieee.org/automaton/robotics/humanoids/aldebaran-robotics-introduces-romeo-finally}{ROMEO}
    d'Aldebaran Robotics
  }

	\cvitem{Mars 2015 -- Sept 2015}
  {Stage : Projet de Fin d'Étude - Ingénieur}
  {\href{https://www.laas.fr/public/fr}{CNRS-LAAS, Toulouse}}
  {
    \textbf{Fusion traitement d'image/détecteur radiofréquence pour la
      détection/localisation d'objects}:\newline
    $\textbullet$ Projet
    \href{http://www.agence-nationale-recherche.fr/Project-ANR-12-CORD-0003}{ANR
      RIDDLE} -- Collaboration CNRS-LAAS, UPS-IRIT, Aldebaran Robotics, CHU
    Toulouse, Magelium\newline
    $\textbullet$ Développement sur \ros\ de briques de détection d'objets via
    traitement d'image
    (\href{http://www.stefan-hinterstoisser.com/papers/hinterstoisser2011linemod.pdf}{LINEMOD} 
    \& \href{http://wiki.ros.org/tabletop_object_detector}{TABLETOP}) (\textbf{C++})
  }

	\cvitem{Fevr 2015 -- Mars 2015}
  {Projet Étudiant Ingénieur}
  {\href{https://www.laas.fr/public/fr}{CNRS-LAAS, Toulouse}}
  {
    \textbf{Algorithme probabilistique de localisation binaurale}:\newline
    $\textbullet$ Projet \href{http://twoears.eu/}{TWO!EARS}\newline
    $\textbullet$ Filtre de Kalman Unscented Multi-Hypothèses (MH-UKF)
    (\textbf{C++})
  }

	\cvitem{Juil 2014 -- Août 2014}
  {Stage: Technicien}
  {\href{http://www.spherea.com/fr}{Cassidian Test \& Services, Colomiers}}
  {
    Stage technicien création liaison PCIe sur FPGA Spartan 6 (VHDL)
  }

	\cvitem{Juil 2013 -- Août 2013}
  {Stage: Ouvrier}
  {Tokyo / Kitakyushu, Japon}
  {
    Ouvrier peintre en bâtiment
  }

	\cvitem{Avri 2012 -- Juin 2012}
  {Stage: Technicien}
  {\href{https://www.gtptech.com/}{GTP Technology}, Labège}
  {
    Instrumentation, acquisition de données (Modbus, UART, RS-485)    
  }
\end{cvtable}

% \newpage
% \makebacksidebar

% \newgeometry{
% 	top=\topbottommargin,
% 	bottom=\topbottommargin,
% 	right=\leftrightmargin,
% 	left=\leftrightmargin
% }

\cvsignature

\end{document}

%%%%%%%%%%%%%%%%%%%%%%%%%%%%%%%%%%%%%%%%%%%%%%%%%
% Commented here in order to save tham as example

%%% - Side Bar

% \profilesection{About Me}
% \aboutme{
% The giant panda is a terrestrial animal and primarily spends its life
% roaming and feeding in the bamboo forests of the Qinling Mountains and in
% the hilly province of Sichuan.
% }

%  \profilesection{Diagrams}
% \begin{sidebarminipage}
%  \chartlabel{Bubble}
% 	\chartlabel{Diagrams}
% 	\chartlabel{with}
% 	\chartlabel{proper}
% 	\chartlabel{overflow}
% 	\chartlabel{protection}
% 	\chartlabel{for}
% 	\chartlabel{labels}
% \end{sidebarminipage}

% \begin{figure}\centering
%  \smartdiagram[bubble diagram]{
% \textcolor{white}{\textbf{Being a}} \\
% \textcolor{white}{\textbf{Panda}}, % center bubble
% \textcolor{black!90}{Eating},
% \textcolor{black!90}{Sleeping},
% \textcolor{black!90}{Rolling},
% \textcolor{black!90}{Playing},
% \textcolor{black!90}{Chilling}
% }
% \end{figure}

% \chartlabel{Wheel Chart}

% \wheelchart{3.7em}{2em}{%
% 20/3em/maincolor!50/Chill,
% 15/3em/maincolor!15/Play,
% 30/4em/maincolor!40/Sleep,
% 20/3em/maincolor!20/Eat
% }

%  \profilesection{Barskills}
% \barskill{\faSkyatlas}{Wearing asian rice hats}{60}
% \barskill{\faImage}{Playing Chess}{30}
% \barskill{\faMusic}{Playing the bamboo flute}{50}

% \profilesection{Memberships}
% \begin{memberships}
%  \membership[4em]{pics/logo.png}{PandaScience.net}
% 	\membership[4em]{pics/logo.png}{Some longer text spanning over more than
% only one line}
% \end{memberships}

% \profilesection{Hard Skills}
% \skill{\faBalanceScale}{Sleeping almost all day}
% \skill{\faSitemap}{Eating a lot of bamboo sprouts}
% \skill{\faGraduationCap}{Relaxing rest of the day}

% \profilesection{Soft Skills}
% \pointskill{\faHome}{Looking Cute}{4}[4]
% \skill[1.8em]{\faCompress}{No need to specify further}
% \pointskill{\faChild}{Chillin' hard}{3}[4]
% \skill[1.8em]{\faCompress}{On a tree}
% \skill[1.8em]{\faCompress}{In the grass}

%%% - Body

% \cvsection{section}
% \cvsubsection{Subsection}
% \begin{cvtable}
% 	\cvitem{<dates>}{<cv-item title>}{<location>}{<optional: description>}
% \end{cvtable}

% \cvsection{cvitem}
% \cvsubsection{Multi-line with longer description}
% \begin{cvtable}
% 	\cvitem{date}{Description}{location}{Some longer and more detailed
% 		description, that takes two lines of space instead of only one.}
% 	\cvitem{date}{Description}{location}{Some longer and more detailed
% 		description, that takes two lines of space instead of only one.}
% 	\cvitem{date}{Description}{location}{Some longer and more detailed
% 		description, that takes two lines of space instead of only one.}
% \end{cvtable}

% \cvsubsection{One-line without description}
% \begin{cvtable}
% 	\cvitem{Award}{One-line description}{Sponsor}{}
% 	\cvitem{Award}{One-line description}{Sponsor}{}
% 	\cvitem{Award}{One-line description}{Sponsor}{}
% \end{cvtable}

% \cvsection{cvitemshort}
% \cvsubsection{One-line}
% \begin{cvtable}
% 	\cvitemshort{Key}{Some further description}
% 	\cvitemshort{Key}{Some further description}
% 	\cvitemshort{Key}{Some further description}
% \end{cvtable}

% \cvsubsection{Multi-line with longer description}
% \begin{cvtable}
% 	\cvitemshort{Key}{Some further description. Can fill even more than
% 		only one single line while still keeping the correct indendation level.}
% 	\cvitemshort{Key}{Some further description. Can fill even more than
% 		only one single line while still keeping the correct indendation level.}
% 	\cvitemshort{Key}{Some further description. Can fill even more than
% 		only one single line while still keeping the correct indendation level.}
% \end{cvtable}

% \cvsection{cvpubitem}
% \begin{cvtable}
% 	\cvpubitem{Publication title}{Authors}{Journal}{Year}
% 	\cvpubitem{Publication title}{Authors}{Journal}{Year}
% 	\cvpubitem{Publication title that is spanning over multiple lines and still
% 		does not look too bad}{Authors}{Journal}{Year}
% \end{cvtable}





% Local Variables:
% TeX-engine: xetex
% End:
