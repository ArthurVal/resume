% FortySecondsCV LaTeX template
% Copyright © 2019-2020 René Wirnata <rene.wirnata@pandascience.net>
% Licensed under the 3-Clause BSD License. See LICENSE file for details.
%
% Please visit https://github.com/PandaScience/FortySecondsCV for the most
% recent version! For bugs or feature requests, please open a new issue on
% github.
%
% Contributors
% ------------
% * ifokkema
% * Bertbk
% * Hespe
% * esben
%
% Attributions
% ------------
% * fortysecondscv is based on the twentysecondcv class by Carmine Spagnuolo
%   (cspagnuolo@unisa.it), released under the MIT license and available under
%   https://github.com/spagnuolocarmine/TwentySecondsCurriculumVitae-LaTex
% * further attributions are indicated immediately before corresponding code


%-------------------------------------------------------------------------------
%                             ADDITIONAL PACKAGES
%-------------------------------------------------------------------------------
\documentclass[
	a4paper,
	% showframes,
	% vline=2.2em,
	% maincolor=cvgreen,
	% sidecolor=gray!50,
	% sectioncolor=cvgreen,
	subsectioncolor=cvblue!70,
	% itemtextcolor=black!80,
	% sidebarwidth=0.4\paperwidth,
	% topbottommargin=0.03\paperheight,
	% leftrightmargin=20pt,
	% profilepicsize=4.5cm,
	% profilepicborderwidth=3.5pt,
	% profilepicstyle=profilecircle,
	% profilepiczoom=1.0,
	% profilepicxshift=0mm,
	% profilepicyshift=0mm,
	% profilepicrounding=1.0cm,
	% logowidth=4.5cm,
	% logospace=5mm,
	% logoposition=before,
]{fortysecondscv}

% improve word spacing and hyphenation
\usepackage{microtype}
\usepackage{ragged2e}

% uncomment in case you don't want any hyphenation
% \usepackage[none]{hyphenat}

% take care of proper font encoding
\ifxetexorluatex
	\usepackage{fontspec}
	\defaultfontfeatures{Ligatures=TeX}
	\newfontfamily\headingfont[Path = fonts/]{segoeuib.ttf} % local font
\else
	\usepackage[utf8]{inputenc}
	\usepackage[T1]{fontenc}
\fi

% \usepackage[sfdefault]{noto} % use noto google font
\usepackage[sfdefault]{ClearSans}

% enable mathematical syntax for some symbols like \varnothing
\usepackage{amssymb}

% bubble diagram configuration
\usepackage{smartdiagram}
\smartdiagramset{
	% default font size is \large, so adjust to harmonize with sidebar layout
	bubble center node font = \footnotesize,
	bubble node font = \footnotesize,
	% default: 4cm/2.5cm; make minimum diameter relative to sidebar size
	bubble center node size = 0.4\sidebartextwidth,
	bubble node size = 0.25\sidebartextwidth,
	distance center/other bubbles = 1.5em,
	% set center bubble color
	bubble center node color = maincolor!70,
	% define the list of colors usable in the diagram
	set color list = {maincolor!10, maincolor!40,
      maincolor!20, maincolor!60, maincolor!35, maincolor!50},
	% sets the opacity at which the bubbles are shown
	bubble fill opacity = 0.8,
}


\usepackage{enumitem}

\newcommand{\cea}{\href{http://www.cea.fr/}{CEA}}
\newcommand{\ros}{\href{http://www.ros.org/}{R.O.S.}}
\newcommand{\ice}{\href{https://zeroc.com/products/ice}{ICE}}
\newcommand{\tango}{\href{https://www.tango-controls.org/}{TANGO}}
\newcommand{\hl}[1]{\textbf{\textcolor{orange}{#1}}}

%-------------------------------------------------------------------------------
%                            PERSONAL INFORMATION
%-------------------------------------------------------------------------------
%% mandatory information
\input{personal.tex}

% your name
\cvname{Arthur VALIENTE}
% job title/career
\cvjobtitle{Expert C/C++/Python\\[0.2em] Instrumentation \& Contrôle}

%% optional information
% profile picture
% \cvprofilepic{pics/profile.png}
% logo picture
% \cvlogopic{pics/logo_txt.png}

% NOTE: ordering in sidebar will mimic the following order
% date of birth
% \cvbirthday{\today}

% short address/location, use \newline if more than 1 line is required
\cvaddress{\MyAddress}

% phone number
\cvphone{\MyPhoneNumber}

% personal website
% \cvsite{https://pandascience.net}

% email address
\cvmail{\MyEMail}

% pgp key
% \cvkey{4096R/FF00FF00}{0xAABBCCDDFF00FF00}

% any other custom entry
% \cvcustomdata{\faGithub}{https://github.com/ArthurVal}

%-------------------------------------------------------------------------------
%                              SIDEBAR 1st PAGE
%-------------------------------------------------------------------------------
% add more profile sections to sidebar on first page
\addtofrontsidebar{
	% include gosquare national flags from https://github.com/gosquared/flags;
	% naming according to ISO 3166-1 alpha-2 country codes
	\graphicspath{{pics/flags/}}

	% social network accounts incl. proper hyperlinks
	\profilesection{Réseaux sociaux}
		\begin{icontable}{1.5em}{1em}
			\social{\faLinkedin}
        {https://www.linkedin.com/in/arthur-valiente-1866937b/}
				{/arthur-valiente-1866937b}
			\social{\faGithub}
				{https://github.com/ArthurVal}
				{/ArthurVal}
		\end{icontable}

	\profilesection{Langues}
		\pointskill{\flag{FR.png}}{Français}{4}[4]
		\pointskill{\flag{GB.png}}{Anglais \scriptsize{(TOEIC 935/990)}}{3}[4]
    %% \pointskill{\flag{ES.png} \flag{JP.png}} {Espagnol/Japonais}{1}[4]
        % \pointskill{\flag{}}{Japonais}{1}

  \profilesection{Compétences}
  \begin{figure}\centering
    \smartdiagram[bubble diagram]{
      \textcolor{white}{\large \textbf{Ingénieur}} \\
      \textcolor{white}{\large \textbf{Logiciel}}\\ % center bubble
      \textcolor{white}{\large \textbf{Embarqué}}, % center bubble
      % --
      \textcolor{black!90}{\large Systèmes}\\
      \textcolor{black!90}{\large Distribués}\\
      \textcolor{black!90}{\ros, \ice}\\
      \textcolor{black!90}{\tango},
      % --
      \textcolor{black!90}{\large{Emacs}}\\
      \textcolor{black!90}{\large{\faCodeBranch GIT}}\\
      \textcolor{black!90}{Matlab}\\
      \textcolor{black!90}{Simulink},
      % --
      \textcolor{black!90}{\large{\faLinux}}\\
      \textcolor{black!90}{\large Linux}\\
      \textcolor{black!90}{\large
        \href{https://www.yoctoproject.org/}{Yocto}},
      % --
      \textcolor{black!90}{\large Progr. :}\\
      \textcolor{black!90}{Meta-, OOP,}\\
      \textcolor{black!90}{Fonctionnelle},
      % --
      \textcolor{black!90}{\large C/C++}\\
      \textcolor{black!90}{\large Python}\\
      \textcolor{black!90}{\large CMake}\\
      \textcolor{black!90}{\faTerminal\ \large{Bash}},
      % --
      \textcolor{black!90}{\large{\faMicrochip} \href{https://en.wikipedia.org/wiki/System_on_a_chip}{SoC}}\\
      \textcolor{black!90}{ARM}\\
      \textcolor{black!90}{DSP}
    }
  \end{figure}
  \begin{sidebarminipage}
    \chartlabel[orange]{\href{https://github.com/google/googletest}{GoogleTest}}
    \chartlabel[orange]{\href{https://github.com/google/benchmark}{Google benchmark}}
    \chartlabel[orange]{\href{https://eigen.tuxfamily.org/index.php?title=Main_Page}{Eigen3}}
    \chartlabel[orange]{\href{https://www.boost.org/}{Boost}}
    \chartlabel[orange]{\href{https://abseil.io/}{Abseil}}
    %% \chartlabel{\href{https://opencv.org/}{OpenCV}}
    %% \chartlabel{\href{https://www.qt.io/}{Qt}}
    \chartlabel[orange]{\href{https://docs.pytest.org/}{pytest}}
    \chartlabel{Doxygen}
    \chartlabel{Jira}
    \chartlabel{Gitlab}
    \chartlabel{Automatique}
    \chartlabel{Robotique}
    %% \chartlabel{\href{https://fr.wikipedia.org/wiki/R\%C3\%A9gulateur_PID}{PID}}
    %% \chartlabel{\href{https://fr.mathworks.com/help/physmod/sps/ref/rstcontroller.html}{RST}}
    %% \chartlabel{\href{https://fr.wikipedia.org/wiki/Commande_LQ}{LQR}}
    %% \chartlabel{\href{https://en.wikipedia.org/wiki/Model_predictive_control}{MPC}}
    \chartlabel{Génie Électrique}
  \end{sidebarminipage}

  \sidesection{Un peu de moi}
  \aboutme{
    \hl{Curieux}, j'aime apprendre/découvrir le fonctionnement des choses. En
    plus de mon \hl{pragmatisme}, mon goût pour la \hl{résolution de problèmes}
    difficiles, ma \hl{patience} et mon rêve d'exploration, l'ingénieurie est
    pour moi la meilleure façon de \hl{partager} et faire profiter de mes
    compétences pour le bien commun.
  }
}


%-------------------------------------------------------------------------------
%                              SIDEBAR 2nd PAGE
%-------------------------------------------------------------------------------
\addtobacksidebar{

}



%-------------------------------------------------------------------------------
%                         TABLE ENTRIES RIGHT COLUMN
%-------------------------------------------------------------------------------
\begin{document}

\makefrontsidebar

\cvsection{Formation}
\begin{cvtable}[1.5]
	\cvitem{2012 -- 2015}
         {Ingénieur : \hl{Génie Électrique et Automatique}}
         {\href{http://www.enseeiht.fr/fr/index.html}{INP-ENSEEIHT, Toulouse}}
         {
           Commande, Décision et Informatique des Systèmes Critiques\\
           Développement des Systèmes Informatiques Critiques
         }
  \cvitem{2010 -- 2012}
         {D.U.T. Mesures Physiques}
         {\href{http://iut-meph.ups-tlse.fr/}{I.U.T Paul Sabatier, Toulouse}}
         {
           Techniques Instrumentales
         }

	\cvitem{2010}
         {Baccalauréat S - Science de l'Ingénieur}
         {\href{http://alexis-monteil.entmip.fr/}{Lycée Alexis Monteil, Rodez}}
         { }
\end{cvtable}

\cvsection{Expériences professionnelles}
% \cvsubsection{\textbf{Emploi}}
\begin{cvtable}[2]
  \cvitem
      {Depuis 05/2025}
      {Ingénieur Dév. Logiciel Embarqué}
      {\href{https://www.erems.fr/fr/}{EREMS, Flourens}}
  {
    Développeur logiciel embarqué
    %% : \newline $\bullet$ {
    %%   Création de tests unitaires LFC .. TODO
    %% }
  }
  \cvitem
      {De 11/2024 \newline À 05/2025}
      {Ingénieur de Recherche}
      {\href{https://www.laas.fr/public/fr}{CNRS-LAAS, Toulouse}}
  {
    Développeur logiciel au sein de l'équipe de recherche GEPETTO
    %% : \newline $\bullet$ {
    %%   Création de tests unitaires LFC .. TODO
    %% }
  }

  \cvitem
      {De 10/2022 \newline À 09/2024}
      {Ingénieur Dév. Logiciel Embarqué}
      {\href{https://fr.wikipedia.org/wiki/Airbus_Defence_and_Space}{Airbus DS, Toulouse}}
  {
    Développeur \hl{logiciel embarqué spatial (R\&D)}:
    \newline $\bullet$ {
      Débug/Investigation sur SoC
      \href{https://nanoxplore.org/index.php/product/ng-ultra/}{NG-Ultra}
      (\textbf{2x ARM R52 + FPGA})
    }
    \newline $\bullet$ {
      Spec/Développement SoC
      \hl{
        \href{https://www.xilinx.com/products/silicon-devices/acap/versal-ai-core.html}
             {Xilinx Versal}
      } (\textbf{D-BUS, C et C++})
    }
  }

  \cvitem
      {De 06/2021\\À 10/2022}
      {Ingénieur Logiciel Contrôle Accélérateur}
      {\href{https://www.esrf.fr/}{ESRF, Grenoble}}
  {
    Responsable de la \hl{maintenance et de l'évolution} des logiciels en charge
    du contrôle de l'accélérateur de particule:
    \newline $\bullet$ {
      Contact principal avec le groupe \hl{'Power Supplies'}
    }
    \newline $\bullet$ {
      Système redondant d'alimentations électriques
    }
    \newline $\bullet$ {
      Développement de fonctionnalités pour le contrôle d'orbite
    }
    \newline $\bullet$ {
      \hl{Système de contrôle distribué} ($\approx$ 23000 processus)
      (\textbf{\tango})
    }
  }

  \cvitem
      {De 01/2021\\À 05/2021}
      {Ingénieur informatique}
      {\href{https://www.capgemini.com/service/digital-services/digital-engineering-and-manufacturing-services/}{Capgemini-DEMS,
      Toulouse}}
  {
    \hl{Product Owner} du Work Package \hl{'Navigation autonome'} du
    \href{https://www.capgemini.com/fr-fr/solutions/5g-water-clean-bot/}{5G
      Water Clean Bot} (\textbf{SCRUM})
  }

  \cvitem
      {De 03/2020\\À 06/2020}
      {Ingénieur informatique}
      {\href{https://www.polytechnique.edu/fr/le-laboratoire-leprince-ringuet-llr}{École
      Polytechnique, Palaiseau}}
  {
    \textbf{\href{http://polywww.in2p3.fr/-cms-45-?lang=fr}{Projet CMS}}:
    Spécification fonctionnelle logiciel de contrôle (\textbf{UML})
  }

  \cvitem
      {De 07/2018\\À 12/2019}
      {Ingénieur de Recherche}
      {\cea-DRF-IRFU, Saclay}
  {
    Responsable du \hl{banc de test distribué} du \hl{logiciel embarqué spatial} de la
    caméra ECLAIRs (Détection de sursauts gamma):
    \newline $\bullet$ {
      \hl{\href{http://www.svom.fr/}{Projet SVOM}} -- Collaboration
      \cea\ - \href{http://www.irap.omp.eu/}{IRAP} -
      \href{https://cnes.fr/fr}{CNES} - CNSA
    }
    \newline $\bullet$ {
      Développement de l'\hl{infrastructure de test distribuée} (2xZedboard, PC,
      plateforme embarqué ECLAIRs) (\textbf{C++}, \textbf{Python})
    }
    \newline $\bullet$ {
      Création des scripts de tests unitaires
      (\href{https://en.wikipedia.org/wiki/Software_requirements_specification}{SRS}
      \href{https://en.wikipedia.org/wiki/Software_verification_and_validation}{V\&V},
      \textbf{Python})
    }
    \newline $\bullet$ {
      CCSDS PUS - Spacewire - Middleware
      \href{https://zeroc.com/products/ice}{\textbf{ICE}} -
      \href{https://www.plm.automation.siemens.com/global/en/products/polarion/}{Polarion}
    }
  }

  \cvitem
      {De 01/2017\\À 12/2017}
      {Ingénieur de Recherche}
      {\cea-DRT-LIST, Saclay}
  {
    Responsable logiciel pour applications robotiques:
    %% \newline $\bullet$ {
    %%   \hl{\href{https://en.wikipedia.org/wiki/Simultaneous_localization_and_mapping}{SLAM}}
    %%   hybride topologique pour robot mobile (Lidar / IMU / Caméra RBG-D / ...)
    %%   (\textbf{C++})
    %% }
    \newline $\bullet$ {
      \textbf{\href{https://www.designspot.fr/portfolio/face/}{Projet FACE}}: En
      charge de la création du prototype de \hl{démonstration multi
        \href{https://en.wikipedia.org/wiki/System_on_a_chip}{System-On-Chip}
        distribué} (Renesas R-Car H3, Nvidia Tx2, Kalray MPPA Bostan) pour
      le véhicule autonome
    }
    \newline $\bullet$ {
      Développement de 'layers' \hl{Linux embarqué Yocto} pour R-Car H3
    }
  }

	\cvitem
      {De 07/2015\\À 12/2015}
      {Ingénieur d'Étude}
      {\href{https://www.irit.fr/?lang=fr}{UPS-IRIT, Toulouse}}
  {
    Développement démonstration robotique sur robots
    \href{http://www.willowgarage.com/pages/pr2/overview}{PR2} et
      \href{https://spectrum.ieee.org/automaton/robotics/humanoids/aldebaran-robotics-introduces-romeo-finally}{ROMEO}
    %% \newline$\bullet$ Algorithme de détection d'objets et intéractions Homme-Robot
    %% \newline$\bullet$ Projet \href{http://www.agence-nationale-recherche.fr/Project-ANR-12-CORD-0003}{ANR RIDDLE}
  }
\end{cvtable}

\cvsubsection{\textbf{Stages | Projets}}
\begin{cvtable}[2]
  \cvitem{6 mois 2015}
  {
    Fusion traitement d'image / radiofréquence\newline
    \hl{\ros - DSP} - Projet
    \href{http://www.agence-nationale-recherche.fr/Project-ANR-12-CORD-0003}{ANR
      RIDDLE}
  }
  {\href{https://www.laas.fr/public/fr}{CNRS-LAAS, Toulouse}}
  {}

  \cvitem{2 mois 2015}
  {
    Localisation binaurale de source sonore\newline
    \hl{\href{https://en.wikipedia.org/wiki/Kalman_filter\#Unscented_Kalman_filter}{Filtre de Kalman UKF}} - Projet \href{http://twoears.eu/}{TWO!EARS}
  }
  {\href{https://www.laas.fr/public/fr}{CNRS-LAAS, Toulouse}}
  {}

  \cvitem{2 mois 2014}
  {Technicien lien \hl{PCIe FPGA} (VHDL)}
  {\href{http://www.spherea.com/fr}{Cassidian Test \& Services, Colomiers}}
  {}

  \cvitem{2 mois 2013}
  {Ouvrier peintre en bâtiment}
  {\hl{Tokyo / Kitakyushu, Japon}}
  {}

  %% \cvitem{3 mois 2012}
  %% {Technicien instrumentation, acquisition de données (\hl{Modbus, UART, RS-485})}
  %% {\href{https://www.gtptech.com/}{GTP Technology}, Labège}
  %% {}
\end{cvtable}

% \newpage
% \makebacksidebar

% \newgeometry{
% 	top=\topbottommargin,
% 	bottom=\topbottommargin,
% 	right=\leftrightmargin,
% 	left=\leftrightmargin
% }

% \cvsignature

\end{document}

%%%%%%%%%%%%%%%%%%%%%%%%%%%%%%%%%%%%%%%%%%%%%%%%%
% Commented here in order to save tham as example

%%% - Side Bar

% \profilesection{About Me}
% \aboutme{
% The giant panda is a terrestrial animal and primarily spends its life
% roaming and feeding in the bamboo forests of the Qinling Mountains and in
% the hilly province of Sichuan.
% }

%  \profilesection{Diagrams}
% \begin{sidebarminipage}
%  \chartlabel{Bubble}
% 	\chartlabel{Diagrams}
% 	\chartlabel{with}
% 	\chartlabel{proper}
% 	\chartlabel{overflow}
% 	\chartlabel{protection}
% 	\chartlabel{for}
% 	\chartlabel{labels}
% \end{sidebarminipage}

% \begin{figure}\centering
%  \smartdiagram[bubble diagram]{
% \textcolor{white}{\textbf{Being a}} \\
% \textcolor{white}{\textbf{Panda}}, % center bubble
% \textcolor{black!90}{Eating},
% \textcolor{black!90}{Sleeping},
% \textcolor{black!90}{Rolling},
% \textcolor{black!90}{Playing},
% \textcolor{black!90}{Chilling}
% }
% \end{figure}

% \chartlabel{Wheel Chart}

% \wheelchart{3.7em}{2em}{%
% 20/3em/maincolor!50/Chill,
% 15/3em/maincolor!15/Play,
% 30/4em/maincolor!40/Sleep,
% 20/3em/maincolor!20/Eat
% }

%  \profilesection{Barskills}
% \barskill{\faSkyatlas}{Wearing asian rice hats}{60}
% \barskill{\faImage}{Playing Chess}{30}
% \barskill{\faMusic}{Playing the bamboo flute}{50}

% \profilesection{Memberships}
% \begin{memberships}
%  \membership[4em]{pics/logo.png}{PandaScience.net}
% 	\membership[4em]{pics/logo.png}{Some longer text spanning over more than
% only one line}
% \end{memberships}

% \profilesection{Hard Skills}
% \skill{\faBalanceScale}{Sleeping almost all day}
% \skill{\faSitemap}{Eating a lot of bamboo sprouts}
% \skill{\faGraduationCap}{Relaxing rest of the day}

% \profilesection{Soft Skills}
% \pointskill{\faHome}{Looking Cute}{4}[4]
% \skill[1.8em]{\faCompress}{No need to specify further}
% \pointskill{\faChild}{Chillin' hard}{3}[4]
% \skill[1.8em]{\faCompress}{On a tree}
% \skill[1.8em]{\faCompress}{In the grass}

%%% - Body

% \cvsection{section}
% \cvsubsection{Subsection}
% \begin{cvtable}
% 	\cvitem{<dates>}{<cv-item title>}{<location>}{<optional: description>}
% \end{cvtable}

% \cvsection{cvitem}
% \cvsubsection{Multi-line with longer description}
% \begin{cvtable}
% 	\cvitem{date}{Description}{location}{Some longer and more detailed
% 		description, that takes two lines of space instead of only one.}
% 	\cvitem{date}{Description}{location}{Some longer and more detailed
% 		description, that takes two lines of space instead of only one.}
% 	\cvitem{date}{Description}{location}{Some longer and more detailed
% 		description, that takes two lines of space instead of only one.}
% \end{cvtable}

% \cvsubsection{One-line without description}
% \begin{cvtable}
% 	\cvitem{Award}{One-line description}{Sponsor}{}
% 	\cvitem{Award}{One-line description}{Sponsor}{}
% 	\cvitem{Award}{One-line description}{Sponsor}{}
% \end{cvtable}

% \cvsection{cvitemshort}
% \cvsubsection{One-line}
% \begin{cvtable}
% 	\cvitemshort{Key}{Some further description}
% 	\cvitemshort{Key}{Some further description}
% 	\cvitemshort{Key}{Some further description}
% \end{cvtable}

% \cvsubsection{Multi-line with longer description}
% \begin{cvtable}
% 	\cvitemshort{Key}{Some further description. Can fill even more than
% 		only one single line while still keeping the correct indendation level.}
% 	\cvitemshort{Key}{Some further description. Can fill even more than
% 		only one single line while still keeping the correct indendation level.}
% 	\cvitemshort{Key}{Some further description. Can fill even more than
% 		only one single line while still keeping the correct indendation level.}
% \end{cvtable}

% \cvsection{cvpubitem}
% \begin{cvtable}
% 	\cvpubitem{Publication title}{Authors}{Journal}{Year}
% 	\cvpubitem{Publication title}{Authors}{Journal}{Year}
% 	\cvpubitem{Publication title that is spanning over multiple lines and still
% 		does not look too bad}{Authors}{Journal}{Year}
% \end{cvtable}





% Local Variables:
% TeX-engine: xetex
% End:
